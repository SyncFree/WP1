A mathematical representation of this use case is represented below.
\begin{enumerate}
	\item $M$ is the total number of \glspl{dc} and $i$ identifies one of the \glspl{dc}, $M \in \mathbb{Z}_{+}$ and $i \in \{1,\dots, M\}$.

	\item $C$ is the total number of clients and $c$ represents a client in the system, $c \in \mathbb{Z}_{+}$ and $c \in \{1,\dots, C\}$.

	\item $Balance = \{ \langle Crsytals, n1 \rangle, \langle Euro, n2 \rangle \}$ maps each currency $curr \in \{Crystals, Euro\}$ to an amount $n1, n2 \in \mathbb{Z}$. $B_{ci} \in B$ keeps the balance and $\widetilde{B}_{ci} \in B$ keeps the consumed amounts of currencies by a client $c$ in \gls{dc} $i$, where $B$ denotes the set of all $Balance$ maps.

	We define the operations $\oplus$ and $\ominus$ on Balance maps $b, b' \in B$ such that: $b \oplus \langle amount \in Z, curr \in Curr \rangle = b'$ where $b'[curr] = b[curr] + amount$,  ~  $b'[c] = b[c]  ~  iff  ~  c != curr$ and $b \ominus(amount \in R, curr \in Curr) = b'$ where $b'[curr] = b[curr] - amount$,  ~  $b'[c] = b[c]  ~  iff  ~  c != curr$. We overload these operations such that they can subtract not only a tuple but a set of tuples (defined in a balance) from another balance: $b \oplus b'$ and $b \ominus b'$ where $b, b' \in B$.  
	
	\item The tuple $Voucher = \langle id, cost, spec \rangle$ defines a voucher where $id \in \mathbb{VID}$  is the voucher id, $cost \in \mathbb{Z}_{+} \times Curr$ and $spec \in Strings$ is the details of the voucher. $V_{ci} \in V$ is a multiset of vouchers of a client $c$ kept in \gls{dc} $i$, where V denotes the set of all vouchers. Similarly, $\widetilde{V}_{ci}$ is the multiset of consumed vouchers a client $c$.
	\textcolor{red}{(Added: The cost of the voucher (Can the cost be specified in terms of different currencies?))}
	
	\item The tuple $Trans = \langle id, ts, type, args, ops \rangle$ defines a transaction where $id \in \mathbb{TID}$ is the unique transaction id, $ts \in \mathbb{Z}_{+}$ is the timestamp, $type \in TTypes$ and $args \in Args$ is the data required for the transaction type.  $ops$ is a list of operations $o \in Ops = \{ purchasedVouc \times V, \; purchasedVCurr \times \mathbb{Z}_{+} \times Curr, \\ consumedVouc \times \mathbb{Z}_{+}, \; consumedVCurr \times \mathbb{Z}_{+} \}$ defines an operation performed in a transaction. \textcolor{red}{(Added: The list of ops in the transaction.)}
 	
	\item The tuple $W_{ci} = \langle B_{ci}, \widetilde{B}_{ci}, V_{ci}, \widetilde{V}_{ci}, T_{c} \rangle$ defines the wallet of a client $c$ kept in \gls{dc} $i$. A wallet keeps the purchased currencies, consumed currencies, purchased set of vouchers, consumed set of vouchers and the list of (not yet archived) transaction logs $T_{c}$ of a client. $|T_{c}| <= MaxTSize$ since the transaction logs in a wallet should be archived when the cardinality of the transactions reaches to the max size.

	\item The net balance of a client can be obtained by subtracting the consumed amounts of currencies from the balance of a client $c \in C$ in \gls{dc} $i$. The net amounts of all currencies should be non-negative.
	\begin{multline}  \label{eq:balance_virtual_wallet}
		\langle curr, n \rangle \in B'_{ci} = B_{ci} \ominus \widetilde{B}_{ci} \implies \\
		n \ge 0 ~ \forall ~  curr \in Curr, ~  c \in \{1,\dots, C\},  ~  i \in \{1,\dots, M\} 
	\end{multline}

	\item The net set of vouchers of a client can be obtained by subtracting the consumed vouchers from the gained vouchers of a client $c \in C$ in \gls{dc} $i$, as shown in Equation \ref{eq:balance_virtual_wallet1}, where $\setminus$ is the multiset difference operator.
	\begin{equation} \label{eq:balance_virtual_wallet1}
		V'_{ci} = V_{ci} \setminus \widetilde{V}_{ci} \\
	\end{equation}
  
	\item The consumed set of vouchers should be a subset of gained vouchers of a client $c$.
	\begin{equation}  \label{eq:balance_virtual_wallet2}
		\widetilde{V}_{ci} \subseteq V_{ci}   
	\end{equation}
		
	\item An operation $o \in Ops$ maps $W_{ci} = \langle B_{ci}, V_{ci}, T \rangle \rightarrow W'_{ci} = \langle B'_{ci}, V'_{ci}, T' \rangle$ as follows: 
		
	$o = \langle purchasedVouc, v, curr \rangle$ where $v=\langle id, cost, spec \rangle \in V, ~ curr \in Curr$:
	
	\textcolor{red}{Additional parameter curr (which currency to use for purchasing? Can we buy a voucher using Crystals?) Does "purchased vouc" reduces balance? }
	\begin{multline} \label{eq:purchaseItem_virtual_wallet}
		 B_{ci}[curr] > cost \implies \langle B_{ci}, \widetilde{B}_{ci}, V_{ci}, \widetilde{V}_{ci}, T \rangle ~ \overset{o}{\rightarrow} \\
	    ~ \langle B_{ci}, \widetilde{B}_{ci} \oplus \{cost, curr\}, V_{ci} \cup \{v\}, \widetilde{V}_{ci}, T' \rangle \\
	    where ~ v = \langle id, \langle cost, curr \rangle, spec \rangle \in V.
	\end{multline}
	
	$o = \langle purchasedVCurr, amount, currency \rangle$ where $amount \in \mathbb{Z}_{+}$ and $currency \in Curr$:
	\begin{multline} \label{eq:purchaseCurr_virtual_wallet}
	    \langle B_{ci}, \widetilde{B}_{ci}, V_{ci}, \widetilde{V}_{ci}, T \rangle ~ \overset{o}{\rightarrow} \\
	    ~ \langle B_{ci} \oplus \{amount, curr\}, \widetilde{B}_{ci}, V_{ci}, \widetilde{V}_{ci}, T' \rangle
	\end{multline}
	
	$o = \langle consumedVouc, v \rangle$ where $v=\langle id, cost, spec \rangle  \in V$:
	\begin{multline} \label{eq:consumeVouc_virtual_wallet}
		 v \in V_{ci} \implies 
	    \langle B_{ci}, \widetilde{B}_{ci}, V_{ci}, \widetilde{V}_{ci}, T \rangle ~ \overset{o}{\rightarrow} \\
	     ~ \langle B_{ci}, \widetilde{B}_{ci}, V_{ci}, \widetilde{V}_{ci} \cup \{v\}, T' \rangle 	
	\end{multline}
	
	$o = \langle consumedVCurr, amount \rangle$ where $amount \in \mathbb{Z}_{+}$:
	\begin{multline} \label{eq:consumeVCurr_virtual_wallet}
	    \langle B_{ci}, \widetilde{B}_{ci}, V_{ci}, \widetilde{V}_{ci}, T \rangle ~ \overset{o}{\rightarrow} \\
	    ~ \langle B_{ci}, \widetilde{B}_{ci} \oplus \langle cost, curr \rangle, V_{ci}, \widetilde{V}_{ci}, T' \rangle 	
	\end{multline}
	
	In all these formulas, $T_{c}'=T_{c} \cup \{ \langle id, ts, type, args, o \rangle \}$, assuming the operation $o$ is the only operation performed in transaction $T_{c}$. The operations are applied in their order of appearance in the list of operations in the transaction.
\end{enumerate}