\section{Introduction}
\gls{ec} is one key element in the concept for a contemporary replicated database, presented in \cite{Shapiro2009a}, \cite{Saito2005a}, that takes into account the CAP theorem \cite{Gilbert2002a} . In fact, the database comprises data traversing the cloud together with data residing on devices, not yet if ever, going to move to another storage which is here referred to as the edge of the network. \gls{acid} are the classic requirements for a database. We want to retain these although we realise that they may be relaxed in some areas and they should be avoided in other areas. Atomicity indicates that either the entire transaction can be carried out or it will be rolled back to the state before it was even started. In a large-scale distributed application this criteria cannot be accommodated in a reliable fashion as it results in deadlocks. Consistency cannot be guaranteed as part of the atomic transaction. We are trusting \gls{ec}, which has been a prolific topic of research \cite{shapiro11comprehensive}, \cite{Vogels2009a}, \cite{Saito2005a}, \cite{Baquero1997a}, and provisioning mechanisms for detecting persistent inconsistencies. \gls{ec} is a weaker data consistency which will require a complex background algorithm for reconciling conflicting updates \cite{Terry1995a}. \glspl{crdt} have been designed to solve the need for reconciliation by using \gls{sec} \cite{shapiro11conflictfree}, without the need of complex conflict resolution, roll-backs, or consensus. Also composites of \glspl{crdt} present the same properties, \cite{Deftu2013a}, \cite{shapiro11comprehensive}.