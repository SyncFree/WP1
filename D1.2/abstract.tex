\begin{abstract}
Building reliable distributed systems, given the CAP theorem where partitioning must always be considered, relays in the trade-offs between consistency and availability. \glspl{crdt} define replicated data types with mathematical properties that ensure absence of conflict and confer them \gls{sec}, a form of \gls{ec} which it is a technique of compromise.  Consistency is a property of the data, not the datastore, given that the rules to decide how to synchronise are business decisions. \glspl{crdt} have been created to ensure that we have a computer model for handling data which accommodates data's nature in a world of vast communication and no definite central place of storage. Using an \gls{rdbms} for these type of systems can result in systems where a large amount of the processing is required to circumnavigate the shortcomings of a model that is not fit for purpose. A significant project for mediating this dilemma is \href{https://syncfree.lip6.fr}{SyncFree}.

%For SyncFree we are gathering requirements from a range of large-scale distributed applications which all have been implemented recently or are in the process of being implemented.

%The goal of SyncFree is to enable large-scale distributed applications without global synchronization, by exploiting the recent concept of \glspl{crdt}. \glspl{crdt} allow non-synchronized concurrent updates, yet ensure data consistency. This revolutionary approach maximizes responsiveness and availability; it enables locating data near its users, in decentralized clouds.

%For reference and later validation of the developed tools we are gathering requirements from a number of representative applications. This is the work carried out in \gls{wp1}. \gls{wp1} is divided into two succeeding parts, where the first has just been completed. The first is to describe the requirements in natural language and the second is to transform this description into a formal mathematical language.

This article provides a brief presentation of the natural language requirements, which shows the special nature of the paradigm shift forcing large-scale distributed applications away from \gls{rdbms}.
\end{abstract}